% -*-mode:LaTeX; coding: utf-8;-*-

% Wahl des Ausgabemodus
\documentclass{beamer}           % Präsentation (Farbe mit \pause)
%\documentclass[handout]{beamer} % Handoutversion (s/w und 4up ohne \pause)
%\documentclass[trans]{beamer}   % wie Präsentation, aber ohne \pause


\usepackage{sansmathaccent}
\pdfmapfile{+sansmathaccent.map}

\usepackage[ngerman]{babel}

% Anpassen ans eingestellte Encoding
%\usepackage[latin1]{inputenc}
\usepackage[utf8]{inputenc}
\usepackage{listings}
\usepackage{booktabs}
\usepackage{hyperref}


\newcommand\tab[1][1cm]{\hspace*{#1}}

\usetheme{HN}

% für anderes Logo als fb03:
\setHNlogo{none}

% Navigationselemente verstecken
\beamertemplatenavigationsymbolsempty

% wenn andere Fußzeile gewünscht, auskommentieren und anpassen:
% footline
%
%\setbeamertemplate{footline}[text line]{
%   \begin{beamercolorbox}[wd=0.98\paperwidth]
%      \insertshortauthor:~\insertshorttitle.~-\insertframenumber-
%      %\inserttotalframenumber
%      \hfill \vskip 0.1cm
%   \end{beamercolorbox}
%}
                             


%------------------------------------------------------------------------
% Titelseite
\title[WPP]{Praktikum Projektfach}

\author[Team]{WPP Team}
\institute{Hochschule Niederrhein - Fachbereich Elektrotechnik \& Informatik}
\date{30. Januar 2019}

\begin{document}
\frame[plain]{\titlepage}

%-----------------------------------------------------------
\begin{frame}{Aufbau der Präsentation}
\tableofcontents  
\end{frame}

%-----------------------------------------------------------
% Strukturierung
\begin{frame}{Strukturierung}
	\begin{itemize}
		\item Aufgabe % Thema, Motivation (Akku effizient nutzen) - Can
		\item Algorithmen % Idee, Funktionsweise, Anwendungsfälle (Eignung), Probleme - Einleitung (Can)
		\begin{itemize}
			\item Korellationsmatrix % Can
			\item Linear % Anna
			\item Newton % Anna
			\item Cubic Splines % Anna
			\item Yesterday % Anna
			\item Averaging % Sebastian
			\item Datenbank % Sebastian
		\end{itemize}
		\item Heuristiken % Idee (Auch Quelle, Dateninteraktion), Funktionsweise
		\begin{itemize}
			\item Durchschnittlicher Tages- \& Nachtverbrauch % Anna
			\item Pattern Recognition % Anna
			\item Feiertage % Anna
			\item Jahreszeiten % Anna
		\end{itemize}
		\item Testing \& Bewertung % Diagramme, Bewertung der Ergebnisse (Rating), Einordnung der Algorithmen - Sebastian
		\item Ausblick % (theoretische) Bewertung der Algorithmen, weitere Ideen - Sebastian
	\end{itemize}
\end{frame}

% Actual content
%!TEX root = ./presentation.tex

\begin{frame}{Inhalt}
	\begin{block}{}
		\begin{itemize}
			\item ...
		\end{itemize}
	\end{block}
\end{frame}


\begin{frame}{}
\fontsize{24pt}{12pt}\selectfont
\centering{Vielen Dank für Ihre Aufmerksamkeit}
\end{frame}

\end{document}
